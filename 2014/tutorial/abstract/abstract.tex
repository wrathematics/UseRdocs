\documentclass[11pt, a4paper]{article}
\usepackage{amsfonts, amsmath, hanging, hyperref, parskip, times}
\usepackage[numbers]{natbib}
\usepackage[pdftex]{graphicx}
\hypersetup{
  colorlinks,
  linkcolor=blue,
  urlcolor=blue,
  citecolor=blue
}

\let\section=\subsubsection
\newcommand{\pkg}[1]{{\normalfont\fontseries{b}\selectfont #1}} 
\let\proglang=\textit
\let\code=\texttt 
\renewcommand{\title}[1]{\begin{center}{\bf \LARGE #1}\end{center}}
\newcommand{\affiliations}{\footnotesize\centering}
\newcommand{\keywords}{\paragraph{Keywords:}}

\setlength{\topmargin}{-15mm}
\setlength{\oddsidemargin}{-2mm}
\setlength{\textwidth}{165mm}
\setlength{\textheight}{250mm}



\usepackage{xcolor}
\usepackage{xspace}

\definecolor{pbdgrn}{HTML}{005700}
\definecolor{pbdrd}{HTML}{ab0000}
\definecolor{pbdylw}{HTML}{ab7e00}
\definecolor{pbdblu}{HTML}{2b74ec}
\newcommand{\pbdR}{%
\textbf{\color{pbdgrn}{p}\color{pbdrd}{b}\color{pbdylw}{d}\color{pbdblu}{R}}%
\xspace}





\begin{document}
\pagestyle{empty}

\title{Programming with Big Data in R}

\begin{center}
  {\bf Drew Schmidt$^{1,^\star}$, George Ostrouchov$^{2,1,\dagger}$}
\end{center}

\begin{affiliations}
1. Joint Institute for Computational Sciences, University of Tennessee 
\\[-2pt]
2. Oak Ridge National Laboratory \\[-2pt]
$^\star$%
\href{mailto:schmidt@math.utk.edu}{schmidt@math.utk.edu}, 
$^\dagger$%
\href{mailto:ostrouchovg@ornl.gov}{ostrouchovg@ornl.gov} 
\\
\end{affiliations}

% We interpret big data literally to mean that its size requires
% parallel processing because it does not fit in the memory of a
% single multicore machine or because we need to make its processing
% time tolerable.
\vspace{1em}

\paragraph{Overview:} 
The tutorial will introduce attendees to high performance computing
concepts for dealing with big data using R, particularly on large
distributed platforms. We will describe the use of the 
\href{http://r-pbd.org}{``programming with big data in R'' (\pbdR)} 
package ecosystem by presenting
several examples of varying complexity. Our packages provide
infrastructure to use and develop advanced parallel R scripts that scale
to \emph{tens of thousands} of cores on supercomputers but also provide simple
parallel solutions for multicore laptops.

Our packages are described in a textbook-style 
\href%
{http://cran.r-project.org/web/packages/pbdDEMO/vignettes/pbdDEMO-guide.pdf}%
{vignette}
associated with
our package 
\href{http://cran.r-project.org/web/packages/pbdDEMO/index.html}{pbdDEMO}. 
This tutorial will follow many of the examples
presented in the document, which we continue to update. We conducted this
tutorial successfully at UseR 2013. However, this year's
tutorial will include numerous updates and improvements, including
demonstrations of some of our new developments and applications completed
since last year.


In this tutorial, we will:
\begin{itemize}
  \item Provide a quick overview of parallel R's capabilities, and discuss how R 
can interface with parallel hardware and HPC libraries.
  \item Discuss the value of profiling, and show off our new profiling package 
pbdPAPI.
  \item Introduce basic MPI programming concepts, and its simplified interface 
via \href{http://cran.r-project.org/web/packages/pbdMPI/index.html}{pbdMPI}.
  \item Discuss parallel data input.
  \item Introduce distributed matrices and distributed matrix methods.
\end{itemize}


\paragraph{Attendee background:} 
We assume intermediate knowledge of R. No prior parallel programming experience 
is necessary. If you wish to follow along on your multicore laptop during the 
tutorial, please install (or check that you have):

\begin{itemize}
  \item R (and Rtools if you are a Windows user)
  \item An MPI library
  \item the \pbdR packages
\end{itemize}

See our 
\href{http://www.r-pbd.org/install.html}{installation instructions}
for details about how to install these requisites for each major platform. 
Please note that we are anticipating having new releases of most of our packages 
just before UseR! 2014, so you may wish to wait until just before the tutorial 
to install these packages.



\paragraph{Workshop Materials:}

Slides and source code for the tutorial will be made available by June
30, 2014 on the \href{http://r-pbd.org/tutorials.html}{\pbdR website}.

% We will have login tokens available for attendees to use the supercomputer 
% Nautilus, a 1024 core SGI system at NICS. Use of this resource is optional even 
% for hands-on purposes, but all pbdR packages are installed there, making 
% following along easier. You will need an ssh client (such as Putty on Windows; 
% Mac/Linux come with ssh by default) in order to make use of this resource. We 
% note that this is our first tutorial use of this resource across the Atlantic. 
% We expect things to go smoothly, but if not, your multicore laptop is the 
% backup.


\paragraph{Presenters:}
Drew Schmidt is a researcher at the University of Tennessee interested in the 
intersection of mathematics, statistics, and high performance computing, and is 
one of the lead developers of the \pbdR project.

George Ostrouchov is Senior Research Scientist at the Oak Ridge National
Laboratory and Joint Faculty Professor at the University of
Tennessee. George's interests are focused on the interaction of high
performance computing and statistics, and he is the architect and lead
of the \pbdR project.

\end{document}

